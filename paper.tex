\documentclass{llncs}

\begin{document}

\title{Untitled CAV Paper}

\author{Alexander Legg\inst{1} 
    \and Leonid Ryzhyk\inst{2}
    \and Nina Narodytska\inst{2}}

\institute{NICTA\thanks{NICTA is funded by the Australian Government as represented by the Department of Broadband,
    Communications and the Digital Economy and the Australian Research Council through the ICT
    Centre of Excellence program.} and UNSW \\
    \email{alexander.legg@nicta.com.au}
    \and Samsung Research}

\maketitle

\begin{abstract}
    This is the abstract
\end{abstract}

\section{Introduction}

Reactive systems are ubiquitous in real-world problems such as circuit design,
industrial automation, or device drivers. Automatic synthesis can provide a
\emph{correct by construction} controller for a reactive system from a
specification.  However, the reactive synthesis problem is 2EXPTIME-complete so
naive algorithms are infeasible on even simple systems.

Reactive synthesis is formalised as a game between the \emph{system} and its
\emph{environment}. In this work we focus on safety games, in which the system
must prevent the environment from forcing the game into an error state. Much of
the complexity of reactive synthesis stems from tracking the set of states in
which a player is winning.

There are several techniques that aim to mitigate this complexity by
representing states symbolically.  Historically the most successful technique
has been to use \emph{Binary Decision Diagrams} (BDDs).  BDDs efficiently
represent a relation on a set of game variables but in the worst case the
representation may be exponential. This means that BDDs are not a
one-size-fits-all solution for all reactive synthesis specifications.

Advances in SAT solving technology has prompted research into its applicability
to synthesis as an alternative to BDDs. One approach is to find sets of states
in CNF \cite{demiurge}. Another approach is to eschew states and focus on
\emph{runs} of the game. Previous work has applied this idea to solving
bounded games \cite{nina} by forming abstract trees of individual runs. In this
paper, we extend this idea to unbounded games by constructing approximate sets
of winning states from abstract trees.

\section{Reactive Synthesis}


\end{document}
