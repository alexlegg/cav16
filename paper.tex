\documentclass{llncs}

\begin{document}

\title{Untitled CAV Paper}

\author{Alexander Legg\inst{1} 
    \and Leonid Ryzhyk\inst{2}
    \and Nina Narodytska\inst{2}}

\institute{NICTA\thanks{NICTA is funded by the Australian Government as represented by the Department of Broadband,
    Communications and the Digital Economy and the Australian Research Council through the ICT
    Centre of Excellence program.} and UNSW \\
    \email{alexander.legg@nicta.com.au}
    \and Samsung Research}

\maketitle

\begin{abstract}
    This is the abstract
\end{abstract}

\section{Introduction}

Reactive systems are ubiquitous in real-world problems such as circuit design,
industrial automation, or device drivers. Automatic synthesis can provide a
\emph{correct by construction} controller for a reactive system from a
specification.  However, the reactive synthesis problem is 2EXPTIME-complete so
naive algorithms are infeasible on even simple systems.

Reactive synthesis is formalised as a game between the \emph{controller} and
its \emph{environment}. In this work we focus on safety games, in which the
controller must prevent the environment from forcing the game into an error
state.  Much of the complexity of reactive synthesis stems from tracking the
set of states in which a player is winning.

There are several techniques that aim to mitigate this complexity by
representing states symbolically.  Historically the most successful technique
has been to use \emph{Binary Decision Diagrams} (BDDs).  BDDs efficiently
represent a relation on a set of game variables but in the worst case the
representation may be exponential. This means that BDDs are not a
one-size-fits-all solution for all reactive synthesis specifications.

Advances in SAT solving technology has prompted research into its applicability
to synthesis as an alternative to BDDs. One approach is to find sets of states
in CNF \cite{demiurge}. Another approach is to eschew states and focus on
\emph{runs} of the game. Previous work has applied this idea to realizability
of bounded games \cite{nina} by forming abstract representations of the game.
In this paper, we extend this idea to unbounded games by constructing
approximate sets of winning states from abstract trees.

\section{Reactive Synthesis}

A \emph{safety game}, $G = \langle X, L_c, L_u, \delta, I, E, \rangle$,
consists of a set of state variables, sets of controllable and uncontrollable
label variables, a transition relationship $\delta : (X, L_c, L_u) \to X$, an
initial state, and an error state. The \emph{controller} and \emph{environment}
players choose controllable and uncontrollable labels respectively and the game
proceeds according to $\delta$. 

An \emph{run} of a game $(x_0, c_0, u_0), (x_1, c_1, u_1) \dots (x_n, c_n,
u_n)$ is a chain of state and label pairs of length $n$ s.t.  $x_{k+1}
\leftarrow \delta(x_k, c_k, u_k)$. A run is winning for the controller if $x_0
= I \land \forall i \in \{1..n\} (x_i \neq E)$. In a bounded game of rank $n$
all runs are restricted to length $n$, whereas unbounded games consider runs of
infinite length.

A \emph{controller strategy} $\pi^c : X \to L_c$ is a mapping of states to
controllable labels. A controller strategy is winning in a bounded game of rank
$n$ if all runs $(x_0, \pi^c(x_0), u_0), (x_1, \pi^c(x_1), u_1) \dots (x_n,
\pi^c(x_n), u_n)$ are winning. Bounded \emph{realizability} is the problem of
determining the existence of such a strategy for a bounded game.

An \emph{environment strategy} $\pi^e : (X, L_c) \to L_u$ is a mapping of
states and controllable labels to uncontrollable labels. A bounded run is
winning for the environment if $x_0 = I \land \exists i \in \{1..n\} (x_i = E)$
and an environment strategy is winning for a bounded game if there exists a run
$(x_0, c_1, \pi^e(x_1, c_1)), (x_1, c_1, \pi^e(x_1, c_1)) \dots (x_n, c_n,
\pi^e(x_n, c_n))$ that wins for the environment. Safety games are zero sum,
therefore the existence of a controller strategy implies the nonexistence of an
environment strategy and vice versa.

\subsection{Abstract Game Trees}

A set of runs can be symbolically represented by a tree of labels. Each edge of
the tree is either a \emph{fixed} valuation of controllable or uncontrollable
variables, or it is \emph{unfixed} denoting any values are possible. Also, the
tree must alternate between controllable and uncontrollable edges to reflect
the alternating turns of the game. The entire set of runs of a bounded game of
rank $n$ is symbolically represented by a tree of depth $2n$ and width 1
populated by unfixed edges. Reducing the set of runs in a game forms an
abstract game, which can be represented symbolically by an \emph{abstract game
tree}.

A strategy is equivalent to the set of all runs with the player's labels
obeying the strategy mapping $\pi$. Therefore, a strategy can also be
represented by a tree. Relaxing the restriction of the strategy mapping allows
for a \emph{partial strategy} in which multiple labels are now possible for a
single state.

A strategy or partial strategy can also be thought of as an abstract game. A
partial strategy for the controller is a restriction only on the controllable
labels in the game. So if the environment can not win in the abstract game
equivalent to the controller's partial strategy, then all strategies allowable
by that partial strategy must be winning.


\end{document}
